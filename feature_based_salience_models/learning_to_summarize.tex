\subsection{Model 2: Learning-to-Summarize}

The \sap{} model has two primary shortcomings on the stream summarization
task. First, by processing sentences in hourly batches, we potentially 
incur latency penalties if important information enters the stream
at the start of the hour.
Ideally we would process new documents as soon as possible in order 
to minimize the negative effects of latency. Second, the salience
regressors were statically trained without exploration. If we had 
some sort of exploration during training, we could take advantage of 
features between previous sentence selection decisions and the current
state of the update summary (something we didn't model in \sap).

In order to train with exploration, we need either many possible trajectories
(i.e. document streams and associated sentence extraction labels) for all
of the plausibly good summaries or some tractable oracle summarizer 
that we can use to complete a trajectory from any partially completed prefix
of extraction decisions. Since we lack even one ground truth trajectory, we
opt to use an oracle summarizer, which we refer to as the oracle policy, 
in keeping with the learning-to-search literature.



%Getting around these limitations poses some severe challenges. Using
%dynamic features means that naive training will require gold reference
%sentence extract sequences and that naively training a sentence salience model
%with these will under explore the space of plausable update summaries,
%meaning that at test time errors are likely to snowball over time.

The oracle policy $\oraclePolicy$ has knowledge of what nuggets, 
if any, are present in each sentence in the document stream, and it's behavior
is quite simple: it processes the document stream sequentially and when it 
encounters a sentence with a novel nugget, it adds that sentence to the update
summary.


As we stated before, training only on a single oracle run over document stream
would be sub-optimal because the oracle is perfect and it would finish 
recovering all of the nuggets quite quickly and then do nothing for
the remainder of the stream. In practice, our learned model is likely to
make mistakes, missing the first few appearances of a nugget but hopefully
recovering them as repetition in the stream makes them more likely to be 
selected. Only following the oracle's first best pass would not help us learn
to recover from errors.

To make better use of the oracle, we adopt the locally optimal learning to
search (LOLS) regime \citep{chang2015learning}, one of a family of 
learning-to-search algorithms. In order to formally describe the algorithm, we
first introduce some notation. We define the stream summarization task as a 
Markov decision process. We observe a sequence of states $\lsState_\lsStep \in 
\lsStateSpace$ which correpsond to observing the first $\lsStep$ sentences in 
the stream along with the first $\lsStep-1$ extraction decisions. A policy 
$\policy \in \policySpace$ maps states to
actions where the action space $\lsActionSpace = \{0, 1\}$ contains two 
actions with $0$ indicating ``skip the current sentence'' and $1$ 
indicating ``extract the current sentence.'' A deterministic transition
function $\fdef{\lsTrans}{\lsStateSpace \times \lsActionSpace}{\lsStateSpace}$.
maps state-action pairs $(\lsState_\lsStep, \lsAction_\lsStep)$ to the next 
state $\lsState_{\lsStep + 1}$, i.e. updating the observed
sentences with the next, i.e. $(\lsStep + 1)$-th, sentence from the stream 
and 
updating the prefix of extraction variables with $\lsAction_\lsStep$.
A trajectory $(\lsState_1, \lsAction_1), \ldots, (\lsState_n, \lsAction_n)$
of state-action tuples
implicitly defines an extract label sequence $\lsSaliences \in \{0, 1\}^n$
for a stream of $n$ sentences.
Additionally, we assume possession of a non-negative loss function 
$\lsStrucLoss(\lsSaliences, \lsSaliencesRef)$
which measures the discrepancy between the predicted extract sequence 
$\lsSaliences$ and a reference sequence $\lsSaliencesRef$.
The model policy interacts with states through a feature
map $\fdef{\lsFeats}{\lsStateSpace}{\Rn{d}}$ and a matrix 
of associated feature
weights $\policyParams\in \Rn{2 \times d}$ via $\policy(\lsState_\lsStep)
= \operatorname{arg\; min}_{\lsAction \in \lsActionSpace} 
    \policyParams_\lsAction \cdot \lsFeats(\lsState_\lsStep)$.


In the LOLS regime, model training works by exploring the state space in 
two phases. Consider a case where the document stream consists of $n$ 
sentences. The first phase, called 
the \textit{roll-in}, uses the current model policy $\modelPolicy$ (which
is initially random) to explore $\lsStep$ steps into the stream, i.e. starting
from state $\lsState_1$, we repeatly apply $\modelPolicy$ and take it's 
predicted action, stopping at state $\lsState_\lsStep$. 
In the second phase we complete the trajectory using either the oracle 
$\oraclePolicy$ or $\modelPolicy$ (the probability $\lsChoiceParam$ of using 
$\oraclePolicy$ over $\modelPolicy$ is a hyperparameter). Let 
$\rollOutPolicy$ be the roll-out policy, and let 
$\lsSaliencesRollOut^{(\lsState_\lsStep, \lsAction)}$ be the extract label
sequence implied by the completed trajectory of using $\modelPolicy$
to navigate to $\lsState_\lsStep$, taking action $\lsAction$, 
and then using $\rollOutPolicy$ to complete the trajectory from 
$\lsState_{\lsStep + 1}$.

We also associate with $\lsSaliencesRollOut^{(\lsState_\lsStep, \lsAction)}$
a non-negative cost $\lsCost(\lsState_\lsStep, \lsAction)$ which connects 
global discrepancy of the resulting structured loss to the local action
decisions via the following formula
\[
    \lsCost(\lsState_\lsStep, \lsAction) = 
      \max_{\lsAction^\prime \in \lsActionSpace}
        \lsStrucLoss(
            \lsSaliencesRollOut^{(\lsState_\lsStep, \lsAction)},
            \lsSaliencesRef)
        - 
        \lsStrucLoss(
            \lsSaliencesRollOut^{(\lsState_\lsStep, \lsAction^\prime)},
            \lsSaliencesRef).
\]
The cost of the locally optimal action will be zero, while the 
the locally sub-optimal action will have a cost equal to the difference
between the structured losses associated with the sub-optimal and optimal 
actions as completed by $\rollOutPolicy$.

For each epoch of training, we select a training stream and for 
each sentence in the stream we perform a roll-in/roll-out, collecting
the associated costs at each step into a dataset
$\dataset = \big\{\big(\lsFeats(\lsState_1), \lsCost(\lsState_1, \lsAction_0), \lsAction_0\big), \ldots,
\big(\lsFeats(\lsState_T), \lsCost(\lsState_T, \lsAction_1), \lsAction_1\big) \big\}$.
After computing all costs for a given
stream $\strsents(\query)$, we update $\modelPolicy$ to minimize the mean
squared error of the sampled costs, i.e. we minimize the function
\[ 
  %  \objective\big(\policyParams, \strsents(\query)\big) =
    \objective(\policyParams) =
    \frac{1}{|\dataset|} \sum_{\lsFeats(\lsState_\lsStep), \lsCost(\lsState_\lsStep, \lsAction), \lsAction \in \dataset}  
\Big( \lsCost(\lsState_\lsStep, \lsAction) -  \policyParams_{\lsAction} \cdot \lsFeats(\lsState_\lsStep, \lsAction) \Big)^2 \]
using stochastic gradient descent. 

\begin{figure}
 \begin{algorithmic}[1]
  \Procedure{LolsTraining}{}
   \State Randomly initialize $\policyParams$.
   \Repeat
    \For{$\query \in \Queries$}
     \State $\dataset \gets \varnothing$
     \For{$\lsStep = 1, \ldots, T$}
        \State Roll-in by executing $\modelPolicy$ for $\lsStep$ steps to 
arrive at state $\lsState_\lsStep$.
        \State Set roll-out policy $\rollOutPolicy = \oraclePolicy$ with probability $\lsChoiceParam$, otherwise $\modelPolicy$.
       \For{$\lsAction \in \lsActionSpace$}
            \State Use $\rollOutPolicy$ to complete trajectory to obtain cost
            $\lsCost(\lsState_\lsStep, \lsAction)$
            \State $\dataset \gets \dataset \cup \{ 
            \big(\lsFeats(\lsState_\lsStep), 
        \lsCost(\lsState_\lsStep, \lsAction), \lsAction \big) \}$
        
       \EndFor
     \EndFor
     \State $\policyParams = \policyParams - \gamma \nabla\objective(\policyParams)$
    \EndFor
   \Until{convergence}
  \EndProcedure
 \end{algorithmic}
 \caption{The locally optimal learning to search (LOLS) training algorithm.
 $\gamma$ is a learning rate.}
 \label{alg:lols}
\end{figure}


The full LOLS training algorithm is presented in \autoref{alg:lols}.
The benefit of using $\modelPolicy$ to obtain the roll-in trajectory 
is that the oberved state spaces are more likely to be similar to ones visited
on test data since compounding error in $\modelPolicy$ will also effect the 
states visited at training 
time and thus minimizing train/test distribution mismatch. 
Randomly selecting $\oraclePolicy$ versus $\modelPolicy$ is similarly 
helpful for ensuring that the completed trajectory is somewhat realistic
(i.e. when using $\modelPolicy$) but can also learn the behavior 
of the more perfect oracle. The theoretical motivations of LOLS can be found
in greater depth in \cite{chang2015learning}.


\subsubsection{Oracle Policy and Loss Function}

We use a greedy oracle that selects sentences that contain novel
nuggets. This oracle will achieve an optimal \comp{} score, i.e.
it will obtain every possible novel nugget in the roll-out phase.
However, it will not always achieve the maximum possible $\expectedGain$.
For example, consider the sequence of sentences $s_1, s_2, s_3$, where
nugget $n_1 \in s_1$, $n_2 \in s_2$, and $n_1,n_2 \in s_3$. The greedy oracle,
proceeding sequentially would select sentences $s_1$ and $s_2$, and skip
$s_3$, achieving an $\expectedGain$ of $\frac{|\{n_1, n_2\}|}{|\{s_1, s_2 \}|}
= \frac{2}{2} = 1$. The maximum achievable $\expectedGain$ is obtained by 
skipping
the first two sentences and selecting sentence $s_3$ yielding $\frac{|\{n_1, n_2\}|}{|\{s_3 \}|} = \frac{2}{1} = 2$. In practice we are far from matching
the greedy oracle and so it suffices for now as an aspirational target.

In order to implement a loss function $\lsStrucLoss$ we need a reference
extract label sequence $\lsSaliencesRef$ for which to compare to the one 
obtained by the model policy, $\predSaliences$. We set 
$\lsSaliencesRef_\lsStep = 1$ if the oracle policy would have selected that
sentence given the prefix of predicted extractions $\predSaliences_{1:\lsStep -1}$, i.e. if the $\lsStep$-th sentence contained a nugget not contained in
any previous sentence extracted by $\modelPolicy$, the reference extract would
contain it.

We design our loss function to penalize policies that severely over- or
 under-generate. 
 Given the reference and predicted extract summaries, 
 we define the loss as the complement of
 the Dice coefficient between the decisions,
\begin{align*}
  \lsStrucLoss(\lsSaliencesRef, \predSaliences) = 1 - 2 \times 
    \frac{ \sum_i \lsSaliencesRef_i \predSaliences_i }{ 
    \sum_i \lsSaliencesRef_i + \predSaliences_i }.
\end{align*}

 This encourages not only local agreement between policies (the numerator of
 the second term) but that the learned and oracle policy should generate
 roughly the same number of updates (the denominator in the second term).

\subsubsection{Salience Estimation}
The model policy accesses a state-action tuple through the feature function 
$\lsFeats$, upon which we learn a linear function of the features to predict 
the anticipated cost of skipping or extracting the sentence. We can interpret
the negative cost of selecting a sentence as its salience to the event
query. Since a
state $\lsState_\lsStep$ encodes the first $\lsStep$ sentences and $\lsStep-1$
extraction decisions made while processing the document stream, $\lsFeats$
can represent a broader range of features than a salience model trained 
on static sentence salience judgements. In particular, we have several
similarity features between the current sentence under consideration and
the current state of the update summary. See \autoref{sec:features} for the 
complete list of features. 

Many of our features are helpful for
 determining the importance of a sentence with respect to its document.
 However, they are more ambiguous for determining importance to the event as
 a whole. For example, it is not clear how to compare the document level
 PageRank of sentences from different documents. To compensate for this, we
 leverage two features which we believe to be good global indicators of
 update selection: the summary content probability and the document
 frequency.  These two features are proxies for detecting (1) a good summary
 sentences (regardless of novelty with respect to other previous decisions)
 and (2) when an event is likely to be producing novel content. 
 Therefore, we also 
 compute
  feature conjunctions of all features in \autoref{sec:features} with the nugget 
 probability and document 
 frequency separately, as well as together.

\subsubsection{Data}
We used the official TREC 2015 Temporal Summarization dataset which 
was expanded from the previous year (on which we evaluated the \sap{} model).
We use an expanded version of the stream corpus used in \autoref{sec:sapmodel}
collected by \cite{frank2012building}. 
The event query dataset was also expanded to 44 total queries with coverage 
in the stream stream corpus. Nuggets for the additional events were also 
collected. Additionally, pooled output from the 2015 TS performer submissions
was manually labeled by NIST to obtain nugget relevance judgements. We used 
these judgements to build the nugget probability feature 
(see \autoref{par:nuggetprob}) and to add noisy labels to the corpus.
 

 Because of the large size of the corpus and the limited size of the manual
 annotation pool,
 many good candidate sentences were not manually reviewed.
 Less than 1\% of the sentences in our relevant document streams received
 manual review.  In order to increase the amount of data for training and
 evaluation of our system, we augmented the manual judgements with automatic
 or ``soft'' matches. A separate gradient boosting classifier was trained for
 each nugget with more than 10 manual sentence matches.  Manually matched
 sentences were used as positive training data and an equal number of
 manually judged non-matching sentences were used as negative examples.
 Ngrams (1-5), percentage of nugget terms covered by the sentence,
 semantic similarity of the sentence to nugget were used as features, along
 with an interaction term between the semantic similarity and coverage. When
 augmenting the relevance judgments with these nugget match soft labels, we
 only include those that have a probability greater than 90\% under the
 classifier. Overall these additional labels increase the number of matched
 sentences by 1600\%.
 
% For evaluation, the summarization system only has access to the query and
% the document stream, without knowledge of any nugget matches (manual or
% automatic).





%~\\~\\
%
%While our learning objective is to minimize the expected 
%structured loss $\lsStrucLoss$,
%the model policy makes greedy local local actions. We use a cost function 
%over state-action pairs to connect the global structure discrepancy to
%local action choices by defining the cost function as 
%%$\fdef{\lsCost}{\lsStateSpace\times\lsActionSpace}{\Rnonneg}$
%\[
%    \lsCost(\lsState, \lsAction)
%\]
%
%
%and update a cost-sensitive 
%classifier to predict the 
%
%
%
%~\\
%
%We treat the streaming summarization problem
%as a Markov decision process. At each timestep $t$ we observe a state
%$s_t \in \mathcal{S}$ and $t$-th sentence $x_t \in \mathcal{X}$ from the 
%document stream.
%A policy $\pi : \mathcal{S} \times \mathcal{X} \rightarrow \{0, 1\}$
%maps a state-sentence tuple to an action $a \in \mathcal{A} =\{0,1\}$
%where $a=1$ indicates we extract the sentence and $a=0$ indicates we skip
%the current sentence. The transition function 
%$d : \mathcal{S} \times \mathcal{X} \times \mathcal{A} \rightarrow \mathcal{S}$
%deterministically maps state-sentence-action tuples to the next state. 
%In practice the state contains set of sentences previously extracted and
%other rolling statistics from previously observed sentences in the stream.
%Our training objective is to minimize the expected costs of each action:
%\[ \mathcal{L}(\pi) = \mathbb{E}_{s,x \sim \pi} \left[ c\left(\pi(s,x), s, x\right) \right] \]
%where $c :\mathcal{A} \times \mathcal{S} \times \mathcal{X} \rightarrow \mathcal{A}$ is the cost of taking a given action in a particular state-sentence
%observation.
%
%~\\
%
%~\\
%
%In the LOLS training regime, there are two phases: the 
%\emph{roll-in} and the \emph{roll-out}. The roll-in phase 
%
%at each timestep 
%we alternate between using the oracle $pi^o$ or our current learned policy
%$\hat{\pi}$ to run multiple time steps into the future and get a cost for 
%a hypothetical summary we would have created in the case where we extracted the 
%current sentence or skipped it. We collect these scores for each decision
%and update a regression model to accuractely predict the cost of each action
%given the current sentence and update summary/stream state.
%
%





\subsubsection{Experiments }

 To evaluate our model, we randomly select five events to use as a
 development set and then perform a leave-one-out style evaluation on the
 remaining 39 events.
 Even after filtering,  each training query's document stream is still too
 large to be used directly in our combinatorial search space. In order to
 make training time reasonable yet representative, we downsample each stream
 to a length of 100 sentences. The downsampling is done uniformly over the
 entire stream. This is repeated 10 times for each training event to create a
 total of 380 training streams. In the event that a downsample contains no
 nuggets (either human or automatically labeled) we resample until at least
 one exists in the sample. 

 In order to avoid overfitting, we select the model iteration for each
 training fold based on its performance (in F-measure of $\expectedGain$ and
 $\comp$) on the development set.


 We refer to our ``learning to summarize'' model in the results as 
 \modelLS.
 We compare our proposed model against several baselines and extensions. 

 \paragraph{Cosine Similarity Threshold} One of the top performing systems 
 at TREC 2015 was a heuristic method that only
 examined article first sentences, selecting those that were below a cosine
 similarity threshold to any of the previously selected updates. We
 implemented a variant of that approach using the latent-vector
 representation used throughout this work \citep{guo2012simple}. 
 The development set was used to
 set the threshold. We refer to this model as \modelCos (Team
 \textsc{WaterlooClarke} at TREC 2015).
  
 \paragraph{Salience-biased Affinity Propagation}
 We also compare this model to a simplified varient of the \sap{} model.
where we use the summary
nugget probability feature as the salience estimator (the Gaussian process
models were difficult to scale to the expanded dataset).  The time
 window size, similarity threshold, and an offset for the cluster preference
 are tuned on the development set.
 As in the \modelCos{} model, a
 similarity threshold is used to filter out updates that are too similar to
 previous updates (i.e. previous clustering outputs).    
 We refer to this method as \sap.  

 \paragraph{Learning2Search+Cosine Similarity Threshold} In this model, which we
 refer to as \modelLSCos, we run \modelLS{} as before, but filter the
 resulting updates using the same cosine similarity threshold method as in
 \modelCos. The threshold was also tuned on the development set. 


 \subsubsection{Results}
 \begin{figure*}
    \center
\begin{tabular}{ l | l l l | l l l | l }
   &\multicolumn{3}{c|}{unpenalized}&\multicolumn{3}{c|}{latency-penalized}&\\
   & exp. gain     & comp. & $F_1$ & exp. gain     & comp. & $F_1$ & num. updates\\
    \hline
%\textsc{AP}         
%    & $0.083$            & $0.09$                     & $0.078$ 
%    & $0.079$            & $0.095$                    & $0.077$ 
%    & ~~$20.846^{s}$ \\
\sap      
    & $\mathbf{0.119}^c$ & $0.09$                     & $0.094$ 
    & $0.105$            & $0.088$                    & $0.088$ 
    & ~~~~$8.333$ \\
\modelCos
    & $0.075$            & $0.176^{s}$              & $0.099$ 
    & $0.095$            & $0.236^{s}$              & $0.128^{s}$ 
    & $145.615^{s,f}$ \\
\modelLS       
    & $0.097$            & $\mathbf{0.207}^{s,f}$   & $0.112$ 
    & $0.136^{c}$      & $\mathbf{0.306}^{s,c,f}$ & $0.162^{s}$
    & ~~$89.872^{s,f}$ \\
\modelLSCos
    & $0.115^{c,l}$        & $0.189^{s}$ & ${\bf0.127}^{s,c,l}$
    & ${\bf0.162}^{s,c,l}$ & $0.276^{s}$ & ${\bf0.184}^{s,c,l}$
 & ~~$29.231^{s,c}$ \\
\end{tabular}
\caption{
 Average system performance 
 and average number of updates per event.
 Superscripts indicate significant improvements ($p < 0.05$) between the run and
 competing algorithms using the 
  paired randomization test with the Bonferroni correction for multiple 
  comparisons ($s$: \sap, 
 $c$: \modelCos, $l$: \modelLS, $f$: \modelLSCos). 
}
\label{fig:lsresults}
\end{figure*}


  Results for system runs are shown in \autoref{fig:lsresults}.  On average,
  \modelLS{} and \modelLSCos{} achieve higher \fmeasure{} scores than the baseline
 systems in both latency penalized and unpenalized evaluations. For
 \modelLSCos, the difference in mean \fmeasure{} score was significant compared
 to all other systems (for both latency settings).
 
 \sap{} achieved the overall highest $\expectedGain$, partially because
 it was the tersest system we evaluated. However, only \modelCos{} was
 statistically significantly worse than it on this measure. %Interestingly,
 
 In comprehensiveness, \modelLS{} recalls on average a fifth of the nuggets
 for each event. This is even more impressive when  compared to the average
 number of updates produced by each system (\autoref{fig:lsresults}); while
 \modelCos{} achieves similar comprehensiveness, it takes on average about
 62\% more updates than \modelLS{} and almost 400\% more updates than
 \modelLSCos. The output size of \modelCos{} stretches the limit of the
 term ``summary,'' which is typically shorter than 145 sentences in length.
 This is especially important if the intended application is negatively
 affected by verbosity (e.g. crisis monitoring).

 \begin{figure}
  \center
  \begin{tabular}{ l | l l l |}
    &\multicolumn{3}{c}{latency-penalized}\\
    & exp. gain     & comp. & $F_1$ \\
    \hline
    \small \modelCos{}  & $0.095$ & $\mathbf{0.236}$ & $0.128$ \\
    \small \textsc{LS-FS}  & $0.164$ & $0.220$ & $0.157$ \\
    \small \textsc{LS-Cos-FS} & 
      $\mathbf{0.207}$ & $0.18~~$ & $\mathbf{0.163}$ \\
  \end{tabular}
  \caption{Average system performance. \textsc{LS-FS} and \textsc{LS-Cos-FS} 
           runs are trained and evaluated on first sentences only (like the 
           \textsc{Cos} system). Unpenalized results are omitted for space but
           the rankings are consistent.}
  \label{fig:lsfsresults}
\end{figure}




 Since \modelCos{} only considers the first sentence of each document, it
 may miss relevant sentences below the article's lead. In order to confirm the
 importance of modeling the oracle, we also trained and evaluated the
 \modelLS{} based approaches on first sentence only streams. Figure
 \ref{fig:lsfsresults} shows the latency penalized results of the first
 sentence only runs.  The \modelLS{} approaches still dominate \modelCos{}
 and receive larger positive effects from the latency penalty despite also
 being restricted to the first sentence. Clearly having a model (beyond
 similarity) of what to select is helpful. Ultimately we do much better when
 we can look at the whole document.
  
 \begin{figure}
  \center
  \begin{tabular}{l|l|l|l|l|l}
    & Miss  &      Miss  &          &    &  \\
    & Lead &      Body &         Empty &   Dupl. & Total \\
    \hline
    \small \sap{}      
      & 29.6\% & 68.7\% & ~~1.6\% & 0.1\% & 15,986\\
    \small \modelCos     
      & 17.8\% & 39.4\% & 41.1\% & 1.7\% & 12,873\\
    \small \textsc{LS-FS}      
      & 25.4\% & 71.7\% & ~~2.0\% & 0.9\% & 13,088\\
    \small \textsc{LS-Cos-FS}
      & 27.9\% & 70.8\% & ~~1.0\% & 0.2\% & 15,756\\
    \small \modelLS{}           
      & 19.6\% & 55.3\% & 19.9\% & 5.1\% & 13,380\\
    \small \textsc{LS-Cos}    
      & 24.6\% & 66.7\% & ~~7.5\% & 1.2\% & 11,613\\
  \end{tabular}
  \caption{Percent of errors made and total errors on test set.}
  \label{fig:lserrors}
\end{figure}


  We also performed an error analysis to further understand how each system
 operates.  \autoref{fig:lserrors} shows the errors made by each system on
 the test streams.  Errors were broken down into four categories. \emph{Miss
 lead} and \emph{miss body} errors occur when a system skips a sentence
 containing a novel nugget in the lead or article body respectively. An
 \emph{empty} error indicates an update was selected that contained no nugget.
 \emph{Duplicate} errors occur when an update contains nuggets but none are
 novel. 
 
  Overall, errors of the miss type are most common and suggest future
 development effort should focus on summary content identification.  About a
 fifth to a third of all system error comes from missing content in the lead
 sentence alone.
 
  After misses, empty errors (false positives) are the next largest source of
  error. \modelCos{} was especially prone to empty errors (41\% of its total
  errors). \modelLS{} is also vulnerable to empties (19.9\%) but after
 applying the similarity filter and restriting to first sentences, these
 errors can be reduced dramatically (to 1\%).
  
  Surprisingly, duplicate errors are a minor issue in our evaluation. This is
 not to suggest we should ignore this component, however, as efforts to
 increase recall (reduce miss errors) are likely to require more robust
 redundancy detection. 

