%\begin{figure}
\center
\begin{tabular}{r|ccccc|ccccc}
& \multicolumn{5}{c}{\bfseries Static Features} 
& \multicolumn{5}{c}{\bfseries Dynamic Features} \\
Model
& \rot{90}{1em}{surface}
& \rot{90}{.5em}{query}
& \rot{90}{1em}{lang. mod.} 
& \rot{90}{1em}{nugget likelihood} 
& \rot{90}{1em}{single doc} 
& \rot{90}{1em}{geographic relevance}
& \rot{90}{1em}{temporal relevance}
& \rot{90}{1em}{document frequency}
& \rot{90}{.5em}{stream language models}
& \rot{90}{.1em}{update similarity} \\
\hline
Salience biased AP & X & X & X &   &   & X & X &   &   &   \\
Learning-to-Search & X & X & X & X & X &   & X & X & X & X \\
\end{tabular}
\caption{Feature groups used by each model.}
\label{fig:strfeats}
\end{figure}




\subsection{Features for Sentence Salience Estimation}
\label{sec:features}

The streaming summarization problem is difficult precisely because the context
is constantly shifting. We cannot rely solely on word frequency because
the counts of particular \ngram s will be shifting throughout the document
stream. Instead we compute several groups of sentence features that are
specifically helpful for the streaming nature of the task. 
Because our models were
developed over several years of the TS track, some
features were not used or available to certain models. If a feature
is only used in one model, we list that model's name in parenthesis next
to the feature group. For more details see \cite{kedzie2015predicting}
and \cite{kedzie2016real}.

%We divide our features
%into two groups: static and dynamic features. Static features do not take into
%account previous sentence selection decisions, while dynamic features
%can be used information about current state of the update summary or stream
%behaviour. %\autoref{fig:strfeats} shows which feature groups were used 
%in each of our two approaches. In general, incorporating dynamic features
%into the learning-to-search approach is much easier.


\begin{itemize}
    \item \textbf{Surface Features.} These features 
include sentence length, the average number of capitalized words,
document position, sentence length in words, and the average number of 
named entities.
    \item \textbf{Query Features.} We employ query match features that count frequency of query term matches
in the sentence. We also do a simple query expansion using WordNet 
\citep{miller1995wordnet}
to find  synonyms, hypernyms, and hyponyms for the query \textit{event type}
and compute
a similar term match count with the expansion.
    \item \textbf{Language Model Score.} We compute the 
        average token log likelihood under two 5-gram Kneser-Ney language 
        models \citep{kneser1995improved}, one trained on newswire documents 
        and one trained on a query 
        \textit{event type} specific sub-corpus of Wikipedia.
    \item \textbf{Geographic Relevance (SAP).} We compute several estimates of the
        distance of the events described in a sentence to the 
        real world location of the event that corresponds to the query.
    \item \textbf{Temporal Relevance (SAP).} We compute rolling average \tfidf{} 
        scores for the last 24 hours to capture spikes in event activity.
    \item \textbf{Document Frequency (L2S).} We compute the hour-to-hour
        change in document
        frequency of the document stream to measure periods of heightened
        event activity. 
    \item \textbf{Stream Language Model Score (L2S).} We compute the average 
        unigram probability of several classes of words (non-stopwords, 
        persons,
        places, and locations) whose word counts are updated with each new 
        document in the stream.
    \item \textbf{Update Similarity (L2S).} We compute several measures of similarity
        between the previously extracted sentences and a candidate sentence.
    \item \textbf{Nugget Probability (L2S).} We compute the probability that a
        sentence contains a nugget using a classifier trained on human 
        judgements using \ngram{} features.
    \item \textbf{Single Document Summarization Rankings (L2S).} We compute sentence
        rankings under several unsupervised extractive single document 
        summarization algorithms.
\end{itemize}


%\subsubsection{Static Features}
%\paragraph{Simple Surface and Query Features} 
%%The document stream that we will be operating over is incredibly noisy,
%%full of automatically extracted article text from raw web pages in 
%%dramatically different formatting. As a result, the underlying text is often
%%full of wep page headers, headlines to other stories, and irrelevant link text.
%%We use these simple shallow surface features to characterize 
%%``normal'' newswire sentences.
%%``normal'' . 
%These features 
%include sentence length, the average number of capitalized words,
%document position, sentence length in words, and the average number of 
%named entities. Length and position features have been used previously in
%other learning based models of sentence salience
%\citep{kupiec1995trainable,conroy2001using}.
%To ensure the focus of the update summaries,
%we employ query match features that count frequency of query term matches
%in the sentence. We also do a simple query expansion using WordNet 
%\citep{miller1995wordnet}
%to find  synonyms, hypernyms, and hyponyms for the query \textit{event type}
%and compute
%a similar term match count with the expansion.
%Queries in the TS data are manually labeled with event type;
%see \autoref{fig:eventtypes} for the list of query event types we used. 
%For the \emph{earthquake} query type, for example, we find the following terms:
%``quake'', ``temblor'', ``seism'', and ``aftershock''.
%
%
%\paragraph{Language Model Scores}
%
%We rely on a pair of language models to assess the likelihood of observing
%a given sentence in the document stream. The first language model is 
%intended to be a generic news model, trained on New York Times and 
%Associated Press sections of the Gigaword corpus \citep{graff2003english}. 
%This model serves a similar purpose as the simple surface features in helping
%to identify newswire-like sentences that are likely to contain informative
%content. The second model is query type specific and intended to focus the summarizer
%on sentences from the same domain of the query. For each query type, we 
%constructed an in-domain corpus of Wikipedia news articles belonging to
%categories and pages related to the event type. 
%%The corpus is 
%%semi-automatically constructed: first a set of high-level categories are 
%%selected and then all pages belonging to those initial categories or
%%any sub-categories are automatically added. 
%%For example, the language
%%model for the event type ``earthquake'' is estimated
%%from Wikipedia pages under the category Category:Earthquakes.
%For the actual summarization system features, we use the average token log
%likelihood under each model as a feature.
%%We use the SRILM toolkit \citep{stolke2002srilm} to implement a 5-gram 
%%Kneser-Ney model \citep{kneser1995improved} for both the general and type
%%specific language models. 
%
%\paragraph{Geographic Relevance (SAP)} 
%%The queries in the TS data are all grounded to physical
%%events in the world and we would like to exploit that. For example, while
%%there are sadly many reports of bombings or explosions at any given time 
%%around the world, when summarizing the query ``boston marathon bombing'' 
%%we are only really interested in news from a relatively proscribed 
%%geographic area, in this case, Boston, Massachusets. Implementing 
%%a geographic relevance signal is quite difficult in practice since 
%%we do not have ground truth location data for each event query, and some
%%events like hurricanes can be on the move. 
%We extract all
%location mentions (using a named entity tagger) and look up the lattitude
%and longitude coordinates
%of each mention using the Bing Maps 
%API\footnote{\url{https://www.microsoft.com/en-us/maps/documentation}}.
%Location
%mentions within the same hour are clustered, and the cluster centers are 
%used as the canonical event locations for the query at that time. 
%For each document we compute the median distance of all location mentions
%to the event location cluster. We also compute the distance of the first
%location mention to the event location cluster. Sentences from the document
%share these two distance features.
%
%
%
%%?The disasters
%%?in our corpus are all phenomena that affect some
%%?part of the world. Where possible, we would like
%%?to capture a sentence’s proximity to the event, i.e.
%%?when a sentence references a location, it should be
%%?close to the area of the disaster.
%%?There are two challenges to using geographic
%%?features. First, we do not know where the event is,
%%?and second, most sentences do not contain references
%%?to a location. We address the first issue by
%%?extracting all locations from documents relevant to
%%?the event at the current hour and looking up their
%%?latitude and longitude using a publicly available
%%?geo-location service. Since the documents that are
%%?at least somewhat relevant to the event, we assume
%%?in aggregate the locations should give us a rough
%%?area of interest. The locations are clustered and
%%?we treat the resulting cluster centers as the event
%%?locations for the current time.
%%?The second issue arises from the fact that the
%%?majority of sentences in our data do not contain
%%?explicit references to locations, i.e. a sequence of
%%?tokens tagged as location named entities. Our intuition
%%?is that geographic relevance is important in
%%?the disaster domain, and we would like to take ad
%
%
%\paragraph{Temporal Relevance (SAP)} As we track a query over time, the document
%stream will ebb and flow as the underlying event unfolds. With hurricanes
%for example, forecasters are usually watching a tropical storm as it grows
%and predicitng where it might make landfall. As a result, news about a major
%hurricane grows gradually, peaking after its main contact with land
%and gradually diminishing. An earthquake on the otherhand gives no warning,
%and there is a sudden spike in the query relevant document stream.
%To capture various event time-series shapes we maintain features for 
%sentence level average \tfidf{} scores for the past 24 hours.
%%While we do not
%%do an extensive study of different event type shapes (which would certainly 
%%be very interesting), 
%%We would like to take advantage of spiking or bursty
%%activity in the document stream as a signal for our summarization system.
%%We do this by maintaining inverse document frequency (\idf) values for each 
%%hour of the document stream.
%%We create 24 average \tfidf{} features for each sentence as if it had occurred
%%in each of the previous 24 hours. 
%%I.e. we fix the observed sentence level 
%%term frequencies
%%and rescale them by \idf{} values cached from each of the previous 24 hours;
%%a single feature value is obtained for each hour by averaging the sentence's 
%%bag of rescaled \tfidf{} values.
%% Large changes in any of the individual
%%features should be representative of a spike in the news.
%
%\paragraph{Document Frequency (L2S)} We also compute the hour-to-hour
%percent change in document frequency of the stream. This
%feature helps gauge breaking developments in an unfolding
%event. As this feature is also heavily affected by the daily
%news cycle (larger average document frequencies in the morning
%and evening) we compute the 0-mean/unit-variance of this
%feature using the training streams to find the mean and variance
%for each hour of the day.
%
%\paragraph{Stream Language Models (L2S)}
%We construct several simple unigram language models that we update with 
%each new document that we see in the stream. Using these models 
%we compute the sum, average, and maximum token likelihood
%of the non-stop words
%in the sentence. We compute similar quantities restricted to
%the person, location, and organization named entities.
%
%\paragraph{Update Similarity (L2S)} The average and maximum cosine similarity
%of the current sentence to all previous updates is computed
%under both the \tfidf{} bag-of-words and latent vector
%representation \citep{guo2012simple}. 
%We also include indicator features for when
%the set of updates is empty (i.e. at the beginning of a run) and
%when either similarity is 0.
%
%\paragraph{Nugget Probability (L2S)}
%\label{par:nuggetprob}
%When developing the L2S model, we had access to a small collection of
%human sentence/nugget judgements from the 2014 Temporal Summarization 
%evaluation.
%We used
%these judgements to train an ngram based classifier to predict whether
%the sentence contained a nugget, and we use the probability of a nugget
%given the sentence text as a feature. 
%
%
%\paragraph{Single Document Summarization Features (L2S)}
%Even though we are processing the stream one sentence at a time in the L2S
%model, we still
%have knowledge of the document that a sentence came from and so we can
%compute features that take into account the other sentences in the document.
%We used several unsupervised sentence ranking methods from the single 
%document summarization literature to get several sentence rank features.
%%Inspired be the \textsc{SumBasic} summarizer \citep{nenkova2005impact}, we
%%compute the average and sum of unigram probability of each sentence using 
%%sentence's document unigram distribution. Inpsired by \cite{guo2013updating},
%%for each sentence we compute the average cosine distance to the remaining 
%%sentences in it's enclosing document. Finally, we compute the document
%%level centroid rank \citep{radev2000centroid} and LexRank 
%%\citep{erkan2004lexrank} for each sentence.
%

