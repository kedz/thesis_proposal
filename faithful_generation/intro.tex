Sections~\ref{sec:feature_salience}~and~\ref{sec:deep_learning_salience}
have focused on identifying the most important content for summary inclusion,
while punting somewhat on summary generation -- and for good reason, 
extractive summarization minimizes the burden of creating fluent text. 
Abstractive text generation adds significant challenges to summary creation, 
but its benefits are many: the ability to achieve tighter compression ratios
for space constrained scenarios \citep{fan2017controllable}, the potential
to target different reading levels \citep{margarido2008automatic}
or style \citep{shen2017style}, and more pragmatically, in
an increasingly copy-protected web, abstractive generation may be the only 
legally viable option for content aggregation services 
\citep{kassam2014google}.

With this expressive power comes the danger that the generated text may
misconstrue the source material. Trust in machine learning
models is increasingly being recognized as an important factor in user 
adoption \citep{ribeiro2016should}, and mistakes of this kind will be 
a show stopper for downstream consumers of summarization. 

\begin{figure}
 \centering
 \includegraphics{images/faithful_generation/example1.pdf}
 \caption{Example of faithful generation from structured data. The generator
 is responsible for producing a list (beam) of candidate utterances from the 
 structured data. The recognizer reranks the beam candidates based on the 
 plausibility of the recovered structured data. }
 \label{fig:fgen_example1}
\end{figure}


As a potential solution,
we propose modeling text generation as a two player game between the generator
and recognizer, akin to an auto-encoder \citep{rumelhart1985learning}. 
First, we provide as input to the 
generator some evidence (e.g., raw text or structured data like entries 
from a database). Conditioned on the evidence, the generator must produce 
a list of candidate utterances describing the evidence. The recognizer
reranks the candidates based on the likelihood of predicting the 
correct evidence.
While we will experiment with a variety of loss functions to find something
that works well empirically, we expect in principal to train the generator
to maximize the expected likelihood of evidence under the recognizer.




See \autoref{fig:fgen_example1} for an example 
where we have biographical data (structured data about name, nationality, 
occupation, etc.), and we generate several plausible and implausible 
candidates that are evaluated by the recognizer.
The first candidate is 
 ranked highly because the recognizer can correctly infer that the 
 nationality of Karen Sp\"arck Jones is British (green box). The second 
 candidate is possibly ok because it maximizes entropy over the choice 
 of nationalies (yellow box), i.e. it makes no commitments either way and does
 not produce a non-true statement. The third beam candidate is clearly wrong
 as the recognizer infers that the nationality is American which is a false
 statement according to the true structured data (red box).

In the remainder of this section, we formally define two faithful generation
models, one for generating text from structured data and 
one for generating text from  text data. We conclude by briefly describing some
applications and datasets we will use to evaluate these modls.

\subsection{Related Work}

%In our proposed faithful generation framework, we use one model, the 
%generator, to
%create a text utterance, e.g. a summary of a document or a description
%of a data table, and another model, the recognizer, to answer questions 
%about the underlying document or table, using the utterance as evidence.
%A good utterance, then, is one that allows the recognizer on average to
%correctly answer those questions.

The conceptual underpinnings of faithful generation trace back to two
ideas in the NLP literature. The first is round-tripping in machine translation
\citep{somers2005round,rapp2009back}, i.e. a good translation system should be able to translate
a source language text to a target language text and then back to the source
language with minimal corruption between the original source and the 
back-translated source. \cite{andreas2016reasoning} have recently carried 
this idea 
over to explaining structured prediction tasks, where, e.g., 
the structure is a 
formal plan for a robotic agent; a rational speaker  generates the text
description of the plan that is most likely to lead a rational listener 
to correctly re-interpret the original plan. Faithful generation is similar:
the generator (speaker) 
 encodes an 
input object
(either text or data table) into 
an intermediate text representation, and the recognizer (listener) then tries to 
answer questions about the source using only the intermediate text.
The main difference is that in faithful generation, we are interested not in
a full reconstruction of the source from the intermediate representation, but
rather, the utility of the intermediate representation on a downstream task, 
e.g. question answering.


The other conceptual precedent is that of discriminative reranking 
of $n$-best lists \citep{collins2005discriminative,charniak2005coarse};
a generative model produces the $n$ most likely latent structures, e.g. parses,
and a discriminative model, which does not have the burden of modeling
the observations, rescores this list.  This idea has been applied 
to text generation as well. \cite{wen2015stochastic} and 
\cite{novikova2017e2e} use a 
sequence-to-sequence
model with beam search to generate $n$-best texts from either a 
dialogue plan or data table respectively,
 and then use a separately trained
classifier to downrank the beam candidates that do not correspond to the 
underlying
structured object.
%statements (according to the data table).
We argue that this is insufficient since we might want to consider 
all beam candidates in a downstream task, e.g. a macro content planner stage.
%that maximizes the  coherency of a sequence generated utterances.
For this model to provide linguistically interesting realizations that are 
still
factually true requires expanding the beam size 
which increases the computational and memory demands at test time. 
In our faithful generation framework, we directly discourage the generator from
ever allowing an untrue statement to be kept in the beam. This is done
using the recognizer directly as a learning signal and backpropagating
through the beam search process.

Increasingly, summarization researchers are exploring
 \textsc{Reinforce}-style
policy gradient methods to optimize non-differentiable metrics like 
\textsc{Rouge} \citep{paulus2017deep,arumae2018reinforced,kryscinski2018improving,narayan2018ranking,pasunuru2018multi}.
The most related to our work is that of \cite{arumae2018reinforced} 
and \cite{pasunuru2018multi}.
\cite{arumae2018reinforced} 
learn an extractive summarization model that maximizes
performance of a question-answering model on cloze style questions created
from the reference summaries. In our proposed method, the questions are 
generated from the source document and we use abstractively generated 
summaries as
the input to the question answering model, a significantly harder task.
\cite{pasunuru2018multi} learn an abstractive summarization model that
optimizes the likelihood that the generated summary is entailed by the 
ground-truth reference summary. The entailment likelihood is obtained from
model trained on the SNLI \citep{bowman2015large} and MULTI-NLI 
\citep{williams2018broad} datasets.
This entailment measure is somewhat orthogonal to the faithful generation 
objective, as our proposed
approach directly evaluates the utility of the summary a faithful proxy
for the underlying document, which is the ultimate goal of the 
summarization task.


Most reinforcement learning applied to abstractive summarization uses
one Monte-Carlo sample from their policy distribution to estimate the 
expected reward. We propose to optimize
the entire beam search so that all beam candidates are viable in downstream
tasks. In this way, faithful generation also resembles 
minimum error rate training (MERT) \citep{och2003minimum} 
and minimum Bayes-risk decoding (MBD)
\citep{kumar2004minimum} in that we are reshaping a distribution of $n$-best
beam candidates to optimize the expected value of an evaluation metric.
We also plan on using reward shaping supervision from the recognizer
model to localize reward signals \citep{mnih2014neural} to specific spans of 
a candidate utterance
that are most repsonsible for violating the input document. Localized
reward shaping will help to penalize only the factually incorrect spans
and preserve syntactic and structural choices that are independent of 
such entailment considerations.


Other notable approaches to improving the faithfulness of abstractive 
generation include \cite{guo2018soft} who use a multi-task training objective 
to learn a shared decoder that can alternatively summarize a document,
generate a question about a document, or generate a logically entailed text 
from a document.
The latter task is relevant here, as the authors claim that this entailment
generation objective encourages the decoder to produce only logical entailed
summaries. This claim deserves further scrutiny as their human 
evaluation only asked about \textit{relevance} and \textit{fluency} where
the relevance criteria included topical relevance and redundancy in addition
to factual accuracy, confounding any interpretation of the generated summaries
as being more faithful to the source document.

%~\\
%This is done by learning from 
%Our proposed method is to learn from the recognizer's signals
%during training, i.e. backpropagating throught the beam search process
%so that the beam does not need to to be pruned during test time, i.e. 
%all surviving beam 
%candidates should be factually true.



%Similarly, minimum error rate training (MERT) \citep{och2003minimum} 
%and minimum Bayes-risk decoding 
%\citep{kumar2004minimum}
%rerank $n$-best translations in order to directly optimize an evaluation
%metric like \textsc{Bleu} \citep{papineni2002bleu}.

%Faithiful generation can be recast in the MERT framework by considering
%the evaluation metric to be the recognizer's classification accuracy on
%table data or cloze accuracy on text data. This is perhaps most similar
%to    


