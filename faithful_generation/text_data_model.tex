\subsection{Text Data Model}
For cases where the evidence is not structured data but text, e.g. 
summarization, we can modify the structured data model slightly to obtain 
a workable faithful training regime. Our data will now consist of 
4-tuples $\evidence, \utterance, \bar{\evidence}, \bar{\utterance} \in 
\dataset$ where $\evidence$ is the input text to be summarized, 
$\utterance$ is the reference
abstractive summary, and $\bar{\evidence}, \bar{\utterance}$ are cloze
style questions and answers \citep{taylor1953cloze} respectively. In
a typical cloze question, a passage is given followed by a sentence with a 
missing word; one must provide the correct word to fill in the blank 
based
on evidence from the passage. We can heuristically create cloze style questions
for summarization by selecting input phrases/sentences with high similarity
to the reference summary, and redacting random content words. See 
\autoref{fig:cloze_example} for an example from the CNN/DailyMail dataset.

\newcolumntype{L}[1]{>{\raggedright\let\newline\\\arraybackslash\hspace{0pt}}m{#1}}


\begin{figure}
\centering
\begin{tabular}{|l|L{95mm}|}
    \hline
   \textbf{Input Text} ($x$) & 
The BBC producer Oisin Tymon allegedly struck by Jeremy
Clarkson will not press charges against the ``Top
Gear'' host, his lawyer said $\ldots \ldots \ldots$\\
\hline
 \textbf{Reference Abstract} ($y$) & 
Producer Oisin Tymon will not press charges against Jeremy
Clarkson, his lawyer says.\\
\hline
 \textbf{Cloze Question} ($\bar{x}$) &
 The BBC producer $\blacksquare~\blacksquare$ allegedly struck by Jeremy
Clarkson will not press charges against the ``Top
Gear'' host, his lawyer said. \\
\hline
 \textbf{Cloze Answer} ($\bar{y}$) &   Oisin Tymon \\
\hline
\end{tabular}



%?\begin{figure}
%?\centering
%?\begin{tabular}{L{28mm}|L{47mm}|L{40mm}|L{20mm}}
%?    \small
%?    \textbf{Input Text} (x)
%?    & \small \textbf{Reference Abstract} (y)
%?    & \small \textbf{Cloze Question} (x) 
%?    & \small \textbf{Cloze Answer} (y) \\
%?%\centering $(\evidence)$ & \centering $(\utterance)$ 
%?%& \centering $(\bar{\evidence})$ 
%?%& \begin{center}$(\bar{\utterance})$\end{center} \\
%?\hline
%?\small
%?The BBC producer Oisin Tymon allegedly struck by Jeremy
%?Clarkson will not press charges against the ``Top
%?Gear'' host, his lawyer said.\newline 
%?~~~~~~~~\large{$\vdots\quad\vdots\quad\vdots$} &
%?Producer Oisin Tymon will not press charges against Jeremy
%?Clarkson, his lawyer says. &
%?The BBC producer $\blacksquare~\blacksquare$ allegedly struck by Jeremy
%?Clarkson will not press charges against the ``Top
%?Gear'' host, his lawyer said. & Oisin Tymon \\
%?\end{tabular}
%?
%?\caption{Example Summarization Cloze Question}
%?\label{fig:cloze_example}
%?\end{figure}
%?
%?
%?
%?\begin{figure}
%?\centering
%?\begin{tabular}{L{40mm}|L{40mm}|L{40mm}|L{20mm}}
%?  \centering \textbf{Input Text} 
%?& \centering \textbf{Reference Abstract}  
%?& \centering \textbf{Cloze Question}  
%?& \begin{center} \textbf{Cloze Answer} \end{center} \\
%?\centering $(\evidence)$ & \centering $(\utterance)$ 
%?& \centering $(\bar{\evidence})$ 
%?& \begin{center}$(\bar{\utterance})$\end{center} \\
%?\hline
%?The BBC producer Oisin Tymon allegedly struck by Jeremy
%?Clarkson will not press charges against the ``Top
%?Gear'' host, his lawyer said Friday.\newline 
%?~~~~~~~~\large{$\vdots\quad\vdots\quad\vdots$} &
%?Producer Oisin Tymon will not press charges against Jeremy
%?Clarkson, his lawyer says. &
%?The BBC producer $\blacksquare~\blacksquare$ allegedly struck by Jeremy
%?Clarkson will not press charges against the ``Top
%?Gear'' host, his lawyer said Friday. & Oisin Tymon \\
%?\end{tabular}

\caption{Example Summarization Cloze Question}
\label{fig:cloze_example}
\end{figure}




We modify our recognizer defintion to the following:
an utterance $\utterance$ is \textbf{faithful}
 to the evidence $\evidence$ under a recognizer $\rec$, 
denoted $\faithful(\utterance,\evidence,\rec)$, if
\[ \rec(\bar{\utterance} |\bar{\evidence}, \utterance) >
 \max_{\bar{\utterance}^\prime \in \vocab \setminus \{\bar{\utterance}\}}
\rec(\bar{\utterance}^\prime |\bar{\evidence}, \utterance)
  \quad \textrm{ or } \quad  
  \entropy_{\max} - \entropy_{\rec(\cdot|\bar{\evidence}, \utterance)} < \epsilon\]
where the entropy terms are defined similarly to 
\autoref{sec:struct_data_model} and the recognizer is now modified to map 
cloze question/passage tuples $\bar{\evidence},\utterance$ to probabilties
of a cloze answer
$\bar{\utterance}$. In practice multiple cloze question/answer pairs will
be created for each input/summary pair. Training of the faithful generator
will proceed similarly to the process outlined in \autoref{sec:struct_data_model}.
