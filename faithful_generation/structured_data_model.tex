  \subsection{Data-to-Text Model}
  \label{sec:struct_data_model}

  Our data-to-text model consists of evidence/utterance tuples $(\evidence, \utterance)\in
  \dataset$.
   The evidence $\evidence = \{\evidence_1, \ldots, \evidence_m\}$
   is a sequence of $m$ categorical variables drawn from  
   $\evidenceSpaces{m}$ where $\evidence_i$ is the observed value for the 
   $i$-th field in the structured data, e.g. the $i$-th field could correspond 
   to \textit{occupation} with possible values
   in $\evidenceSpace_i = 
   \{ \textsc{Accountant}, \textsc{Actor},
    \ldots, \textsc{Zoologist} \}$ .
The utterance $\utterance = \{\utterance_1,\ldots, 
\utterance_{|\utterance|}\}$ is a 
  sequence of $|\utterance|$ tokens with each token drawn from a fixed 
  vocabulary $\vocab$. The utterances correspond to natural language 
  realizations of a subset
  of the evidence.

  The first player in our game is the \textit{generator} $\gen$, which can 
  generate a list of $k$ candidate utterances $\utterance^{(1)},\ldots,
  \utterance^{(k)}$. In practice, $\gen$ is a sequence-to-sequence 
  model \citep{bahdanau2014neural} with parameters $\genParams$, and the $k$ 
candidates are produced using beam search.
 The second player, called the \textit{recognizer}, has a mapping
 $\fdef{\rec_i}{\vocab^*}{(0,1)^{|\evidenceSpace_i|}}$ of utterances to
 probabilities over values for each field
 $i \in \Idx{m}$. We say that an utterance $\utterance$ is \textbf{faithful}
 to the evidence $\evidence$ under a recognizer $\rec$, 
denoted $\faithful(\utterance,\evidence,\rec)$, if, for all $i\in \Idx{m},$
\[ \rec_i(\evidence_i|\utterance) >
 \max_{\evidence^\prime \in \evidenceSpace_i \setminus \{\evidence_i\}}
  \rec_i(\evidence^\prime|\utterance) 
  \quad \textrm{ or } \quad  
  \entropy^{(i)}_{\max} - \entropy_{\rec_i(\cdot|\utterance)} < \epsilon\]
where $\entropy^{(i)}_{\max}$ is the maximum entropy for the $i$-th field,
i.e. the entropy of the uniform distribution 
$\log |\evidenceSpace_i|$, and $\entropy_{\rec_i(\cdot|\utterance)}$ 
is the entropy of $\rec$ over the $i$-th field. In English, 
an utterance is faithful to the evidence if the recognizer can correctly
predict the true evidence from the utterance (left statement) or, barring that,
the recognizer cannot infer a value of the evidence with
anything better than random chance (right statement).

We evaluate the degree to which the generator is faithful with the quantity
\[ \objective_\beamDistr(\evidence) = \E_{\utterance \sim p(\cdot|\evidence;\theta)} 
 \Ind{\faithful(\utterance, \evidence, \rec)}, \]
where $p(\cdot|\evidence;\theta) $ is approximated using beam search.
%\propto \frac{\gen(\utterance|\evidence;\genParams)}{|\utterance|}$ 
%is a the generator distribution renormalized by average token
%probability over the
%beam candidates $\utterance^{(1)},\ldots, \utterance^{(k)}$.

We implement the recognizer as a collection of separate neural network 
classifiers $q_i$  with parameters $\recParams_i$ for each data field,
 and fit the parameters  with 
the average log likelihood objective \[ \objective_\rec(\phi) = 
\frac{1}{|\dataset|} \frac{1}{m} \sum_{\evidence, \utterance \in \dataset} \sum_{i=1}^m 
\log \rec_i(\evidence_i|\utterance;\recParams_i).\] After training, the 
recognizer is frozen and we do not update the parameters when fitting the 
generator.

The generator is pre-trained also using the standard maximum log likelihood
objective: \[\objective_\gen(\genParams) = 
\frac{1}{|\dataset|} \sum_{\evidence, \utterance \in \dataset} 
\log \gen(\utterance|\evidence;\genParams).\] To make the generator faithful,
we then maximize the joint objective \[
\frac{1}{|\dataset|} \sum_{\evidence, \utterance \in \dataset} 
\Big[
\log \gen(\utterance|\evidence;\genParams) + 
 \E_{\utterance \sim p(\cdot|\evidence;\theta)} 
 \Ind{\faithful(\utterance, \evidence, \rec)} \Big], 
\]
where the second term results in a \textsc{Reinforce} style gradient
update \citep{edunov2018classical}.


\newcolumntype{L}[1]{>{\raggedright\let\newline\\\arraybackslash\hspace{0pt}}m{#1}}


\begin{figure}
\centering
\begin{tabular}{|l|L{95mm}|}
    \hline
   \textbf{Input Text} ($x$) & 
The BBC producer Oisin Tymon allegedly struck by Jeremy
Clarkson will not press charges against the ``Top
Gear'' host, his lawyer said $\ldots \ldots \ldots$\\
\hline
 \textbf{Reference Abstract} ($y$) & 
Producer Oisin Tymon will not press charges against Jeremy
Clarkson, his lawyer says.\\
\hline
 \textbf{Cloze Question} ($\bar{x}$) &
 The BBC producer $\blacksquare~\blacksquare$ allegedly struck by Jeremy
Clarkson will not press charges against the ``Top
Gear'' host, his lawyer said. \\
\hline
 \textbf{Cloze Answer} ($\bar{y}$) &   Oisin Tymon \\
\hline
\end{tabular}



%?\begin{figure}
%?\centering
%?\begin{tabular}{L{28mm}|L{47mm}|L{40mm}|L{20mm}}
%?    \small
%?    \textbf{Input Text} (x)
%?    & \small \textbf{Reference Abstract} (y)
%?    & \small \textbf{Cloze Question} (x) 
%?    & \small \textbf{Cloze Answer} (y) \\
%?%\centering $(\evidence)$ & \centering $(\utterance)$ 
%?%& \centering $(\bar{\evidence})$ 
%?%& \begin{center}$(\bar{\utterance})$\end{center} \\
%?\hline
%?\small
%?The BBC producer Oisin Tymon allegedly struck by Jeremy
%?Clarkson will not press charges against the ``Top
%?Gear'' host, his lawyer said.\newline 
%?~~~~~~~~\large{$\vdots\quad\vdots\quad\vdots$} &
%?Producer Oisin Tymon will not press charges against Jeremy
%?Clarkson, his lawyer says. &
%?The BBC producer $\blacksquare~\blacksquare$ allegedly struck by Jeremy
%?Clarkson will not press charges against the ``Top
%?Gear'' host, his lawyer said. & Oisin Tymon \\
%?\end{tabular}
%?
%?\caption{Example Summarization Cloze Question}
%?\label{fig:cloze_example}
%?\end{figure}
%?
%?
%?
%?\begin{figure}
%?\centering
%?\begin{tabular}{L{40mm}|L{40mm}|L{40mm}|L{20mm}}
%?  \centering \textbf{Input Text} 
%?& \centering \textbf{Reference Abstract}  
%?& \centering \textbf{Cloze Question}  
%?& \begin{center} \textbf{Cloze Answer} \end{center} \\
%?\centering $(\evidence)$ & \centering $(\utterance)$ 
%?& \centering $(\bar{\evidence})$ 
%?& \begin{center}$(\bar{\utterance})$\end{center} \\
%?\hline
%?The BBC producer Oisin Tymon allegedly struck by Jeremy
%?Clarkson will not press charges against the ``Top
%?Gear'' host, his lawyer said Friday.\newline 
%?~~~~~~~~\large{$\vdots\quad\vdots\quad\vdots$} &
%?Producer Oisin Tymon will not press charges against Jeremy
%?Clarkson, his lawyer says. &
%?The BBC producer $\blacksquare~\blacksquare$ allegedly struck by Jeremy
%?Clarkson will not press charges against the ``Top
%?Gear'' host, his lawyer said Friday. & Oisin Tymon \\
%?\end{tabular}

\caption{Example Summarization Cloze Question}
\label{fig:cloze_example}
\end{figure}



We plan to validate the utility of this approach on two data-to-text tasks. 
The first is a biography generation task using biographical Wikipedia
entries and structured data extracted from the corresponding Wikipedia page's
info box \citep{lebret2016neural}. The second data-to-text task is a 
restaurant description generation task, where the structured data consists
of data about restaurants (types of food offered, location, etc.) 
\citep{novikova2017e2e}.
Additionally, we will perform a human evaluation where we measure the accuracy
of the beam candidates with respect to the table data.


