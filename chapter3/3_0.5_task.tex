~\\
~\\







We propose two novel approaches to the streaming summarization task
using biased affinity propagation (AP) clustering \citep{frey2007clustering}
and learning-to-search (L2S) \citep{daume2009search,chang2015learning}.
These models led to the publication of two papers 
\citep{kedzie2015predicting,kedzie2016real}, in addition to participation 
in the TREC Temporal Summarization tracks where we were the top performer in
2014 and the fourth and fifth place performer in 2015 
\citep{aslam2015trec,aslam2016trec}. 

In the Temporal Summarization track, the grounding scenario was disaster
summarization, where a participant system received a brief query string 
$\query$ describing a natural or man-made disaster, and the system was 
expected to process a time-ordered stream of documents relevant to the query, 
extracting sentences that were likely to contain important facts about the
event. Each query corresponded to a real-life disaster that was significant
enough to have an associated entry in Wikipedia.

For each query, human annotators also collected a reference set of important 
facts, which we refer to as \textit{nuggets}, 
from the revision history of that query's associated Wikipedia page. 
Nuggets consist of a piece of reference text and timestamp for when this piece
of information first appeared in the Wikipedia revision history. 
See \autoref{fig:eventsnuggets} for example queries and nuggets.

\begin{figure}
    \begin{tabular}{p{7.5cm}|p{7.5cm}}
         \textbf{``hurricane sandy''} &\textbf{``boston marathon bombing''} \\
        \hline
        \parbox{7.4cm}{\small
    $[\textrm{10/23 8:20pm}]$ Sandy strengthened from a tropical depression 
    into a tropical storm \\
    $[\textrm{10/23 8:20pm}]$ 2 pm Oct 23 Sandy moving north-northeast at 4 
    knots \\
    $[\textrm{10/23 8:53pm}]$ forecast track uncertain \\
    $[\textrm{10/25 12:20am}]$ In Jamaica damage was extensive
} & \parbox{7.4cm}{ \small
    $[\textrm{04/15 7:31pm}]$ Authorities are investigating a report of two 
    explosions at the finish line of the race. \\
    $[\textrm{04/15 8:29pm}]$ Some Boston Transit system service has been 
    halted.\\
    $[\textrm{04/16 12:37am}]$ Police confirmed another explosion at the JFK
    Library.
}
\end{tabular}
    \caption{Example event queries in bold, and several example nuggets below.
    Nugget timestamps are show in brackets.}
    \label{fig:eventsnuggets}
\end{figure}




We will refer to the set of extracted sentences as the update or extract
summary interchangeably. Similarly, we will use the phrases extract a sentence,
select a sentence, or emit a sentence to mean we add a sentence in the 
document stream to our rolling update summary. If a sentence $\strsent$ 
expresses the same piece of information as a nugget text $\nugget$ 
we say that $\strsent$ contains $\nugget$ or $\nugget \in \strsent$.

Systems are rewarded when they find sentences that contain important and novel
nuggets. Systems are penalized for 
selecting sentences that are irrelevant (i.e. contain no nuggets) or 
contain nuggets already covered by previous updates. 
Latency penalized metrics are also computed where
the importance of a nugget decays over time. E.g. if a system
recovers the nugget ``25 people were reported injured,'' several days
after this fact was first reported, it will receive less credit for it
than the system that emits that nugget an hour after it enters the 
document stream. See \cite{aslam2014trec} for more details on this decay 
factor. Intuitively, latency penalized metrics capture the idea that stale
information in a rapidly evolving disaster is less useful and possibly
distracting.

Streaming summarization is a very hard task compared to single and 
multi-document summarization. In the latter case, the context for the 
summarization is fixed, and the input documents are usually quite 
topically focused, minimizing the prevalence of completely irrelevant 
information. In fact, in most multi-document evaluation settings, the
document collections were manually created leading to very topically
coherent text collections. 
\cite{baumel2016topic} for example found that the DUC
query focused summarization datasets are so on topic that a summarization
system could completely ignore the query and perform just as well as a
query aware system.

\begin{figure}
\begin{subfigure}{.47\textwidth}
\begin{algorithmic}[1] 
\Procedure{OracleSummarize}{}
  \State $\updates \gets \varnothing$ \Comment{Init. update summary.}
  \State $\hat{\nuggets} \gets \varnothing$ \Comment{Init. found nuggets.}
  \For{$\strsent \in \strsents$} 
    \If{$\exists \nugget$ s.t. $\nugget \in \strsent
            \wedge \nugget \notin \hat{\nuggets}$ }
      \State $\mathcal{U} \gets \mathcal{U} \cup \{\strsent\}$ 
      \State $\hat{\mathcal{N}} \gets \hat{\mathcal{N}} \cup \{\nugget \in \strsent\}$ 
    \EndIf
  \EndFor
  \State \Return $\mathcal{U}$
\EndProcedure
\end{algorithmic}
    \caption{Greedy oracle streaming summarization algorithm.}
    \label{alg:ts_greedy_oracle}
\end{subfigure}
~
\begin{subfigure}{.48\textwidth}
\begin{algorithmic}[1] 

  \Procedure{GenericSummarize}{}
  \State $\mathcal{U} \gets \varnothing$ \Comment{Init. update summary.}
  \For{$\strsent \in \mathcal{S}$} 
    \State $\hat{\salience} \gets \operatorname{salience}(\strsent, \query) $
    \State $\similarity \gets \operatorname{similarity}(\strsent, \updates)$
    \State $\extract \gets \operatorname{decide}(\hat{\salience}, \similarity)$
    \If{$\extract = 1$}
        \State $\updates \gets \updates \cup \{\strsent\}$
    \EndIf
  \EndFor
  \State \Return $\updates$
\EndProcedure
\end{algorithmic}
    \caption{Greedy generic streaming summarization algorithm.}
    \label{alg:ts_greedy_generic}
\end{subfigure}
\caption{Oracle and generic greedy stream summarization algorithms. 
$\mathcal{S}$ is a time-ordered set of sentences from the document stream 
for query $\query$.}
\end{figure}




If we had clairvoyant knowledge of which nuggets were contained in each
sentence in the document stream,
this task would be trivial in that the greedy algorithm in 
\autoref{alg:ts_greedy_oracle} would return an approximately optimal summary.
Since we do not know this information at test time, we attempt approximate
the if-statement in line 5 of the oracle by answering the following proxy 
questions:
\begin{enumerate}
    \item How salient is sentence $\strsent$ with respect to query $\query$?
    \item How similar is the sentence $\strsent$ with respect to the set of 
            previously selected updates $\hat{\strsent} \in \updates$?
\end{enumerate}
The generic stream summarization algorithm we aim to implement is presented
in \autoref{alg:ts_greedy_generic}.

The two models we propose to solve these more tractable problems come with
different trade-offs. They both, however, rely on a feature-base 
representation of the stream sentences to make salience and selection 
predictions. In the next sub-section, we describe the different feature
groups before describing each of the models used to perform the streaming
summarization task.
